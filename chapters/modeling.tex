\chapter{Modeling Magnetic Induction System}\label{C:modeling}

% , we opt for model C for being the most adequate in shape and for working in conjunction with the quadcopter. Model A has a radius too small for landing maneuvers. On the contrary, model B has a radius bigger than the quadcopter complicating the coil support. 




% \begin{table}[ht]
% \centering
% \begin{tabular}{|c|c|c|c|c|c|}

% \noalign{\global\arrayrulewidth1pt}
% \hline
% \textbf{Model name}  &   \textbf{Inductance} 	&   \textbf{Resistance} 	&   \textbf{Q factor} 	&   \textbf{Capacitor}  \\
% \hline
% \hline
% % \tablefootnote{The low inductance and resistance values of model A are due to weight restriction. Model A is a 23\% shorter than other models. By winding one turn involved to exceed mass constraint. Bigger diameters required more adhesive fixation.}

% Model A 	& 7.62$\mu$H 	& 0.489$\Omega$   & 	97 		& 31 nF     \\ \hline 
% Model B  	& 12.72$\mu$H 	& 0.619$\Omega$   & 	129 	& 31 nF 	\\ \hline
% Model C 	& 13.26$\mu$H 	& 0.651$\Omega$   & 	127 	& 31 nF 	\\ \hline

% \end{tabular}
% \caption{Theoretical coil calculations for a test frequency of 1 MHz}
% \label{T:theoretical}
% \end{table}

 



