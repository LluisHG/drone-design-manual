\cleardoublepage
\phantomsection
\chapter{Introduction}

\section{Motivation of the Project}

In the last years, our society has experienced a silent, but intense trend towards autonomous electronic devices (e.g. laptops, digital cameras, smartphones, smartwatches, etc.) that we use in our daily life. In most cases, theses devices are powered by batteries, which need to be charged very often. Conventional recharges are made through the use of wires transmitting electrical energy from the generating point to the electrical device, but it has many problems related to alternating electric current while distribution. Even so, the greatest drawback is the lack of mobility. Moreover consumers have become inside a recharging lifestyle leaded by the combination of high-performance handheld electronics with a current battery technology uncapable to satisfy consumers' desire.

This fact motivated us to wonder whether there exist physical principles that could enable wireless powering of these and similar devices. Different technologies allow contactless energy transfer, each of them with their respective pros and cons. Recent investigations made us move towards inductive coupling systems due to its safety, lack of interference, and efficiency at medium ranges. 

\section{Objectives}
The aim of this project is to accomplish a brand new concept of inductive coupling implementing and outfitting the induction system on a nano-quadcopter with restrictive payload capabilities. A flying energy transporter widens the prospects and possibilities of unmanned aerial vehicles in such different fields, as aerospace, biomedical, multisensors, smart farming and robotics applications.

The study and modeling a resonant inductive system permits to design any inductive system by only having the requirements, such as power level or transfer distance. Based on different papers and projects, we decide to design the system with the purpose of transferring power up to 20 cm. This distance will constraint the dimensions of the inductive coils carried by the quadcopter.

One of the important goals of this project is to design and implement the inductive system inside a closed \textit{system} such as the nano-quadcopter. To achieve this, a strict calculation of performance and weight of each circuit should be made in order not to exceed the maximum take-off mass. 

Eventually, it is intended to demonstrate our system by powering a sensor, and also charging a small battery. Depending on the overall circuit efficiency and on the carrying energy, the transfer distance could be increased or reduced.  

\section{Brief History}\label{sec:timeline}

The idea of Wireless Power Transfer (WPT) is almost 200 years old. In 1826 Andre-Marie Ampere developed Ampere's circuital law, which shows the capacity of the electric current to produce a magnetic field. Five years later, Michael Faraday developed in England the Faraday's law of induction describing electromagnetic force can be induced in a conductor by a time-varying magnetic flux. In the United States Joseph Henry, independently to Faraday, discovers the same induced currents.\cite{keynote1}

In 1867 James Maxwell predicted the existence of electromagnetic waves. Twenty years later, the first spark transmitter generated a spark in a receiver that was several meters away from it. The German physicist Heinrich Hertz proved the existence of electromagnetic waves using this example \cite{2009wireless}.

The Serbian American inventor and engineer Nikola Tesla learned of Hertz's work by the following year and began duplicating his experiments.

In 1891, before the electrical-wire grid, Tesla proposed the first WPT theories \cite{meyer} carrying out various wireless transmission and reception experiments though air or matter. But, it was in 1894 when Tesla developed the equipment to wirelessly light incandescent lamps at his New York laboratory. This method used Resonant Inductive Coupling (RIC), which involves tuning two nearby coils to resonate at the same frequency. After this, no significant advances were made for more than 50 years. 

In 1969, Peter Glaser propose a transmission power link from space down the Earth. The project was named \textit{Solar Power Satellite} and it was based in harvesting solar radiation in space using satellites, which would convert it to microwave energy and then transmit it to Earth for use in electrical power systems.

In the early 1970s, experiments with RFID tags, done by the U.S. government \cite{RFID}, began and by the early 2000's the Professor She Yuen developed a charger to provide resonant power transfer for small electronics. 

Recently, in 2007 MIT researchers were able to power a 60 Watt light bulb from a power source while providing forty percent efficiency over distance in excess of two meters using RIC. Until that moment, the maximum transfer distances achieved between transmitter and receiver were on centimeter range scale. This event signified a turning point in WPT systems. In 2009 Sony showed a wireless electrodynamic-induction powered TV set, 60 V over 50 cm. Haier showed a wireless LCD TV at CES 2010 using researched Wireless Home Digital Interface \cite{medical}.

In July 2010 wireless charging technology for portable electronic devices up to 5 $W$ reached commercialization stage through the launch of the \textit{Qi} Standard by the Wireless Power Consortium, now comprising over 135 companies worldwide. Practically, it means that all receivers under \textit{Qi} specification can be supplied by all transmitters, signed with \textit{Qi} Standard, embracing compatibility between different devices.


\section{Category for the Wireless Power Transfer Systems} 

Wireless energy transfer systems, also called wireless power transfer (WPT), basically work by modulating the generated electric, magnetic, or electromagnetic fields to transport power from a transmitter towards a receiver at certain distance.

WPT systems can be cataloged by many ways, for example by the efficiency, power level, operating frequency, transmission distance, and so on. In this project, we classify the category of wireless power transfer systems by the working range. Figure \ref{fig:classification} shows the category.

As above figure shows, there are two basic sorts. They are the near-field transfers and the far-field transfers, since the field propagation behaviour and the consequent propagation losses strongly differ depending on the field region. 

In near-field or nonradiative region, the oscillating electric and magnetic fields are separate \cite{sazonov2014wearable} and power can be transferred via electric fields ($\vec{E}$) by capacitive coupling (electrostatic induction) or via magnetic fields ($\vec{B}$) by inductive coupling (electromagnetic induction) between coils of wire. These fields are not radiative, meaning the energy stays within a short distance of the transmitter and if there is no receiving device within their limited range to couple to, no power leaves the transmitter. 

In radiative or far-field region the electric and magnetic fields are perpendicular to each other and propagate as an electromagnetic wave, such as microwaves, radio or light waves. This part of the energy is radiative, meaning it leaves the antenna whether or not there is a receiver to absorb it. The portion of energy which does not strike the receiving antenna is dissipated and lost to the system.

The boundary between the two kinds of transfers is vaguely defined. For transmitters and receivers in diameters shorter than half of the operating wavelength, the near field is the region within a radius of wavelength ($r<\lambda$), while the far-field is the region out of a radius of two wavelengths ($r>2\lambda$). The middle region between is known as ``transition zone''. For transmitters and receivers in diameter larger than a half-wavelength, the near and far field transfers are defined by the Fraunhofer distance \cite{Balanis}:

  \begin{equation} % {equation*} --> no numerar
    d_f = \frac{2D^2}{\lambda}
  \end{equation}

where $D$ is the dimension of the largest antenna of the power transmitter and the receiver, $\lambda$ is the wavelength of the electromagnetic wave. 

There exist other radiative technologies, such as radio or WiFi, which use the same fields and waves as wireless power transmission systems. In this case, the main goal is to transmit information, so the amount of power reaching in the receiver is unimportant as long as it is enough to achieve a reasonable signal to noise ratio, making the message intelligible.

\hfill \break
\begin{figure}[hb]
\begin{tikzpicture}[  grow'=right,
                      level distance = 5.25cm,
                      sibling distance = 0.75cm
                    ]

\tikzset{edge from parent/.style = {thick, draw, edge from parent fork right},
         every tree node/.style =
            {rectangle,rounded corners,drop shadow,draw,fill = white,minimum width = 2.8cm, minimum height = 1.1cm,text width = 3.4cm,align = center, text = black}}

\Tree 
    [. {WPT\\ Systems} 
        [.{Near-field\\ Transfers}
            [.{Capacitive Coupling} ]
            [.{Inductive Coupling} ]
        ]
        [.{Far-field\\ Transfers}
            [.{Microwave} ]
            [.{Propagating\\ Electromagnetic} ]
            [.{Photo-electricity} ]
        ] 
    ]

\end{tikzpicture}
\caption{Types of wireless power transfer inside the EM spectrum}
\label{fig:classification}
\end{figure}

%--------------

\section{Discussion}\label{sec:discussion}

% The inductive coupling is the only viable power transfer method which met all the requirements in terms of power and efficiency. It also accomplishes the first and one of the most important proposal in this work, to design a small wireless transfer system capable to be carried in a nano quadcopter.

% Discussion Radiative vs. Non-radiative
In radiative techniques $\vec{E}$ and $\vec{B}$ field strength decreases with distance from the source as $1/r^2$ for the radiated power intensity of electromagnetic radiation. However, near-field $\vec{E}$ and $\vec{B}$ strength decreases more rapidly with distance, being proportional to $1/r^3$. We could think this can be a bother when transferring power. And that is true, but this effect is mostly notable when transmitting over long-distance, which is not the priority of the project. This advantage of far-field over near-field techniques lies on the capability to focus electromagnetic radiation by reflection or refraction into beams. To achieve this narrow beams are necessary antennas much larger than the wavelength of the waves, corresponding to frequencies above 1 GHz, in the microwave range or above. Far-field techniques were rapidly refused because of the physical constraints of the transmitter antenna size, discussed on section \ref{subsec:geo}.

% Discussion between Non-radiative 
Seeing all the available technologies scope inside non-radiative techniques it is quite reasonable to guide towards Resonant Magnetic Induction. This power transfer is reminiscent of the usual magnetic induction; however, the usual non-resonant induction is very inefficient for midrange applications which compromise distances from one antenna diameter up to ten times the antenna diameter \cite{Karalis200834}. % 1e6 más eficiente
As opposed to directed electromagnetic radiation, such as lasers, it does not need an uninterrupted line of sight between the source and the device, as well as a sophisticated tracking mechanism when the device changes its position.

In addition, the fact that magnetic fields interact so weakly with biological organisms is also important for safety considerations \cite{TechTalk}. Capacitive coupling was rejected because of safety issues related to the necessity of a high source voltage. 

To summarize, Resonant Magnetic Induction was the unique WPT system which met mostly all requirements. It allows us to transfer power, nearly omni-directional, over midrange distances in a efficient way. Furthermore, this WPT system is irrespective of the geometry of the surrounding space, with low interference and losses into environmental objects. It also accomplishes the first and one of the most important proposal in this work; to design a small wireless transfer system capable to be carried in a nano-quadcopter.


\begin{table}[ht]
\begin{center}
\begin{tabular}{|l|c|c|c|c|}
% \hline
\rowcolor{black!60}
\hline
\color{white}WPT system                    & \color{white}Frequency   & \color{white}Directivity   & \color{white}Range   & \color{white}Efficiency    \\ \hline %\hline
\rowcolor{gray!54}
Capacitive Coupling           & Low Hz$\sim$MHz     & Weak         & Short       & High             \\ \hline
\rowcolor{white}
Inductive Coupling            & Low Hz$\sim$MHz     & Weak         &  Short      & High             \\ \hline
\rowcolor{gray!40}
Propagating Electromagnetic   & Medium MHz$\sim$GHz       & Medium         &  Medium      & Medium             \\ \hline
\rowcolor{white}
Microwave   & High GHz$\sim$THz       & Strong         &  Long      & Low             \\ \hline
\rowcolor{gray!40}
Photo-electricity   & High $>$THz       & Strong         &  Long      & Low             \\ \hline % Light waves

\end{tabular}
\caption{A comparison among the wireless power transfers}
\label{T:types}
\end{center}
\end{table}
