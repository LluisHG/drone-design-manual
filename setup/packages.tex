% ALL NEEDED PACKAGES ARE LOADED HERE


% Generate English ordinal numbers
\usepackage[super]{nth}

% Improved citation handling in LaTeX
\usepackage[sort,nocompress]{cite}

% Graphic elemens 
\usepackage{tikz}

\usetikzlibrary{shapes.geometric, arrows}
\usetikzlibrary{trees}
\usetikzlibrary{shadows}

\tikzstyle{startstop} = [rectangle, rounded corners, minimum width=3cm, minimum height=1cm,text centered, text width=3cm, draw=black, fill=white]
\tikzstyle{io} = [trapezium, trapezium left angle=70, trapezium right angle=110, minimum width=3cm, minimum height=1cm, text centered, text width=2.8cm , draw=black, fill=white]
\tikzstyle{process} = [rectangle, minimum width=3cm, minimum height=1cm, text centered, text width=3cm, draw=black, fill=white]
\tikzstyle{decision} = [diamond, minimum width=2cm, minimum height=1cm, text centered, text width=2cm, draw=black, fill=white]
\tikzstyle{arrow} = [thick,->,>=stealth]

% Use existing qtree syntax for trees in TikZ
\usepackage{tikz-qtree} % Tree diagram

% Draw electrical networks with TikZ
\usepackage{circuitikz} % Electrical circuits

% Control layout of itemize, enumerate, description
\usepackage{enumitem}   % Introduce lists of items

% Customising captions in floating environments
\usepackage[font={small,it},labelfont=bf]{caption}

% Improved interface for floating objects
\usepackage{float}

% Deprecated: Figures divided into subfigures
\usepackage{subfigure}

% Colour control for LaTeX documents / Add colour to LaTeX tables
\usepackage{color, colortbl}

%  A variable-width \parbox command
\usepackage{pbox}

% Access bold symbols in maths mode
\usepackage{bm} % Provide bold maths symbols

% Enhanced support for graphics
\usepackage{graphicx}
\graphicspath{/images/}
\addtolength{\footnotesep}{1mm} % change to 1mm
\interfootnotelinepenalty=10000 % Avoid to break a footnote into two different pages


%% * OPCIONS A CONFIGURAR al \documentclass
%%    - Estat del document: final o draft
%%      NOTA: Draft no inserta les figures i marca només l'espai que
%%      ocupen. També s'indica quan el text sobrepassa els marges.
%%      Draft és molt útil per compilar ràpid el document si no és important
%%      en aquell moment visualitzar les figures.
%%    - Idioma PRINCIPAL del document: catalan, spanish, english, french...

% Multilingual support for LaTeX, LuaLaTeX, XeLaTeX, and Plain TeX
\usepackage[english]{babel}

%%    NOTA: per canviar d'idioma al mig del document usar:
%%          \selectlanguage{nom_idioma}
%%%%%%%%%%%%%%%%%%%%%%%%%%%%%%%%%%%%%%%%%%%%%%%%%%%%%%%%%%%%%%%%%%%%%%%%%%%%%

%%%%%%%%%%%%%%%%%%%%%%%%%%%%%%%%%%%%%%%%%%%%%%%%%%%%%%%%%%%%%%%%%%%%%%%%%%%%%
% 2- CÀRREGA DE PAQUETS ADICIONALS (OPCIONALS)
%%%%%%%%%%%%%%%%%%%%%%%%%%%%%%%%%%%%%%%%%%%%%%%%%%%%%%%%%%%%%%%%%%%%%%%%%%%%%

% Accept different input encodings
\usepackage[utf8]{inputenc}
\DeclareUnicodeCharacter{00A0}{ }

% Mathematical symbols of American Mathematical Society  
\usepackage{amssymb, amsmath, amsfonts}

% Extending the array and tabular environments
\usepackage{array}                      %% El paquet array proporciona eines molt útils a l'hora de fer equacions amb matrius       

% Create tabular cells spanning multiple rows
\usepackage{multirow}                   %% Paquet que permet fer taules fusionant cel·les de files consecutives          

% Allow tables to flow over page boundaries
\usepackage{longtable}                  %% Paquet molt útil en cas de tenir taules molt llargues que ocupin vàries pàgines          

% Numbered cases environment
\usepackage{cases}                      % Two different labeled equations (warning: tiene que ir detras de 'ansmath')

% Permit footnotes in tables
\usepackage{tablefootnote}

% Include PDF documents in LaTeX
\usepackage{pdfpages}

% Easily include nicely syntax highlighted m-code in your LaTeX documents
\usepackage[framed,numbered,autolinebreaks,useliterate]{mcode}


%% Paquet molt útil que permet activar links en el PDF final.
%% * NO OBLIDAR DE CONFIGURAR els quatre primer camps!
% Extensive support for hypertext in LaTeX
\usepackage[
  pdfauthor={Nom Cognoms autor},            % Configurar adientment
  %pdftitle={Treball Fi de Carrera - autor}, % Configurar adientment
  %pdfsubject={Titol del TFC aqui},          % Configurar adientment
  % Modificació respecte a la versió 2.1 - Iván Padilla Montero - Juliol 2014
  pdftitle={Treball Fi de Grau - autor}, % Configurar adientment
  pdfsubject={Titol del TFG aqui},          % Configurar adientment  
  pdfkeywords={keyword1, keyword2, ...},    % Configurar adientment
  pdfcreator={EETAC-UPC}, 
  pdfproducer={LaTeX, dvipdf},
  pdfdisplaydoctitle=true, plainpages=false, linktocpage=true,         
  colorlinks=true, linkcolor=blue,citecolor=blue,urlcolor=blue,
  hyperfootnotes=false, pagebackref=true, pdfpagelabels=true,            % PARA NO PONER NUMERO DE PAGINA EN LA REFERENCIA
  pdfpagemode=UseOutlines,
]{hyperref}
%% NOTA IMPORTANT!:
%% Per tal que hyperef funcioni correctament amb els capitols o seccions no
%% numerats (\chapter*{}), com per exemple introducció, conclusions i bibliografia
%% cal posar les dues comandes seguents ABANS del \chapter*{} en questió


% View the layout of a document
\usepackage{layout}

% Draw a page-layout diagram
\usepackage{showframe}

% Graphics package-alike macros for “general” boxes
\usepackage[export]{adjustbox}


%% Permet arranjar matricialment multiples figures
%% NOTA: afegir aquest paquet DESPRES del hyperref!!
%%       Si no es desitja utilitzar aquest paquet, comentar la linia seguent
%%       i anar TAMBE al fitxer de classe (eetac_tfc_pfc.cls) per substituir: 
%%       \RequirePackage[subfigure]{tocloft}  per  \RequirePackage{tocloft}
% Automates layout when using the subfigure package
\usepackage{subfigmat}     



% UNUSED

% \usepackage{mwe}
% \usepackage{subfigure}
% \usepackage{etoolbox}
% \usepackage[labelfont=bf]{caption}
% \usepackage{subfloat}
% \usepackage{subcaption}
% \usepackage{showframe,subcaption}

%% Permet trencar links URL. 
%% Atenció! afegir aquest paquet DESPRES del hyperref!!
%\usepackage{breakurl} 